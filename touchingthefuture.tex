\documentclass[11pt,a4paper,oneside]{article}
\usepackage{./latexstyles/print}

\title{
	Kooperative Schreiben mit GitHub ---
	Ein wissenschaftsdidaktisches Experiment mit TeilnehmerInnen des Kurses ``Schule und Demokratie'' der Deutschen SchülerAkademie
}

\date{May 31, 2014}

\author{
	Maximilian Held\\
	Bremen International Graduate School of Social Sciences (BIGSSS)
	\and
	Verena Kasztantowicz\\
	Grundschulpädagogik, Humboldt-Universität zu Berlin (HU)
}

\begin{document}

\selectlanguage{ngerman}

\maketitle

Im Rahmen der Deutschen SchülerAkademie wird im Sommer 2014 ein interdisziplinärer Kurs zum Thema ``Schule und Demokratie'' unter unserer Leitung stattfinden.
Zielgruppe sind SchülerInnen im Altern von 16 bis 18 Jahren, die den 3-wöchigen Kurs im Rahmen des Begabtenförderungsprogramms gewählt haben.

\section{Inhaltliche Konzeption des Kurses ``Schule und Demokratie''}

Inhaltlich steht im Zentrum des Kurses ``Schule und Demokratie'' der Widerspruch und die wechselseitige Bedingtheit von inhärenter Gleichheit und Autonomie im menschlichen Zusammenleben \cite{Habermas1999a}.
Ein konstitutiver Konflikt, der sowohl in der Pädagogik als auch in der politischen Theorie anhaltend diskutiert wird.
Es geht dabei einerseits um die Frage, wie Schule, wie Lernen, und letztlich auch menschliche Beziehungen unter der Prämisse von unterschiedlichen Voraussetzungen demokratisch gestaltet werden kann.
Andererseits wird nach der Legitimität kollektiv verbindlicher Entscheidungen in Politik und Gesellschaft gefragt.
Unter dieser Perspektive rücken auch aktuelle Ansätze wie Formen deliberativer Demokratie (Cohen 1989) und inklusiver Pädagogik (Eberwein/Knauer 2009) ins Blickfeld und sollen innerhalb des Kurses schwerpunktmäßig behandelt werden.
Abschließend versucht der Kurs eine radikale Synthese der beiden Bezugsdisziplinen Pädagogik und politischer Theorie, nämlich, dass eine demokratische Schule eine Schule für alle Kinder und Jugendlichen sein müsse im Sinne inklusiver Pädagogik (Feuser 2001) und deliberative Demokratie gleichermaßen der operativen Metapher der Schule folgen solle (Rosenberg 2002).

\section{Wissenschaftsdidaktische Konzeption des Kurses ``Schule und Demokratie''}

Die beiden vertretenen Disziplinen Pädagogik und politische Theorie erfordern nicht nur weitgreifende inhaltliche sondern auch didaktische Brückenschläge, die mithilfe von innovativen Technologien gestalten werden sollen.

Die Basis unserer Kursarbeit --- und man könnte auch behaupten unserer Disziplinen --- ist der Text, das zusammenhängende geschriebene Wort.
Dabei wird die These vertreten, dass nicht allein durch die Textrezeption, sondern erst durch produktiven und kommunikativen Umgang mit der Literatur ein tiefgreifendes Verständnis bei den TeilnehmerInnen angebahnt wird.
So ist auch das Ziel des Kurses *ein* Text, in Form einer gemeinsam geschriebenen Dokumentation, welche abschließend mit den anderen Beiträgen der Akademie veröffentlicht wird.
Ein Text schreibt sich bekanntlich nicht *von allein*, aber am Besten doch allein, so die landläufige Meinung.
Wie kann trotzdem partizipativ an einem Text geschrieben werden?
Wie bekommen trotzdem alle 15 TeilnehmerInnen die Möglichkeit, sich mit Ideen zu inhaltlichen Schwerpunkten, zum Aufbau und der Form der Dokumentation einzubringen?
Diese Fragen, die normalerweise zu verheerenden E-Mailbombardements führen oder unüberschaubare .doc-Sammlungen in den Dropboxen dieser Welt produzieren, haben gerade für die textroutiniertesten Disziplinen, der Geistes- und Sozialwissenschaften, eine besondere Dringlichkeit.
Damit das Chaos nicht vorprogrammiert ist, haben wir uns an einer weiteren Disziplin orientiert, die tagtäglich vor den Problemen von Textproduktion und Zusammenarbeit steht: Programmierer.
In der Informatik werden seit Langem Plattformen zur Herstellung von Software genutzt, Beispiel hierfür sind ...
    % weitere Anbieter?
Wir haben uns für GitHub entschieden, ein kommerzieller Anbieter, der allerdings für den Bildungskontext freie Kapazitäten zur Verfügung stellt.
    % technische Details und Ablauf auf GitHub

Vorteile der Benutzung dieser Software ist die lückenlose digitale Dokumentation und damit auch die Sicherung aller Erarbeitungsschritte und Ergebnisse; die Möglichkeit aller TeilnehmerInnen direkt auf Textstellen (Absätze, Sätze, einzelne Wörter) Bezug zu nehmen, zu kommentieren und Veränderungen direkt im Text vorzunehmen, die von den anderen Autoren wiederum akzeptiert oder abgelehnt werden können.
    % weitere Vorteile

Nur so kann der Kurs selbst seinem inhaltlichen Postulat nach demokratisch legitimen Zusammenwirken innerhalb der Gemeinschaft des Kurses gerecht werden.

%bibliography
	%\bibliographystyle{apsr}
	%\bibliography{held_library}
		%note, these are *not* submodules, but they point to the clones on the local harddrive, so they must be cloned, too -- maybe they should be implemented as submodules in future.
\end{document}
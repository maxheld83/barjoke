% !TEX program = pdflatex

\documentclass
	[
		11pt,
		a4paper,
		oneside,
		ngerman
	]
	{article}
\usepackage{./latexstyles/print}
\addbibresource{../held-library/held_library.bib}
\addbibresource{barjoke_library.bib}
\addbibresource{../vk_library/verenakasztantowicz.bib}

\title{
	Von Open Source lernen, heißt Schreiben lernen ---
	%MH das original ist wohl "von der sowjetunion lernen heisst siegen lernen" http://de.wikipedia.org/?title=Liste_gefl%C3%BCgelter_Worte/V#Von_der_Sowjetunion_lernen_hei.C3.9Ft_siegen_lernen.
		%sollte so aber ok sein, oder?
	Ein wissenschaftsdidaktisches Experiment zum gemeinsamen Schreiben auf GitHub
}

\date{today}

\author{
	\href{http://www.maxheld.de}{Maximilian Held}, PhD Fellow\\
	\href{mailto:mheld@bigsss.uni-bremen.de}{\texttt{mheld@bigsss.uni-bremen.de}}\\
	Bremen International Graduate School of Social Sciences (BIGSSS)\\
	Universität Bremen
	\and
	\href{https://www.erziehungswissenschaften.hu-berlin.de/grundschulpaed/mitarbeiterinnen/lb-deutsch/v.-kasztantowicz}{Verena Kasztantowicz}, Wissenschaftliche Mitarbeiterin\\
	\href{mailto:kasztave@hu-berlin.de}{\texttt{kasztave@hu-berlin.de}}\\
	Erziehungswissenschaften / Grundschulpädagogik \\
	Humboldt-Universität zu Berlin
}

\begin{document}

\maketitle

\begin{abstract}
	Wir berichten über einen bei der Deutschen SchülerAkademie (Bildung \& Begabung GmbH) durchgeführten Kurs zum Thema ``Schule und Demokratie'', bei dem Teilnehmende im Alter von 16--18 Jahren auf der Entwicklerplattform \emph{GitHub} zusammen schreiben und lernen.

	Konventionen und Werkzeuge aus der Open Source Softwareentwicklung ermöglichen es auch Studierenden der Geistes- und Sozialwissenschaften, kooperativ und in weiten Teilen selbstorganisiert an \emph{einem gemeinsamen Text} zu arbeiten.
	Durch viele Revisionen, wechselseitige Rückmeldungen von Peers und Lehrenden stärken Studierende ihren schriftlichen Ausdruck im Sinne einer \emph{Deliberate Practice}.

	Schließlich erlernen Studierende ein gängiges Softwareprogramm und reflektieren über den Einsatz digitaler Technologie für akademische Zusammenarbeit.
\end{abstract}

Die Basis der sozial- und geisteswissenschaftlichen Didaktik --- wie der zu vermittelnden Disziplinen --- ist der Text, das zusammenhängende geschriebene Wort.
Lernende erwerben nicht allein durch die Textrezeption, sondern erst durch \emph{produktiven} und \emph{kommunikativen} Umgang mit der Literatur ein tiefgreifendes Verständnis.
Erst wenn ein sozial- oder geisteswissenschafliches Konzept in einem \emph{eigenen}, an einen \emph{Leser} gerichteten Text verarbeitet ist, ist es hinreichend verstanden.
	%MH spiele hier an auf die soziale Funktion von Texten, das müssten wir eigentlich auchnochmal aufboren, hast Du mal erwähnt zum Thema schriftspracherwerb.

Herkömmliche universitäre Lehre, ob als Seminar oder als Vorlesung, bietet hier traditionell nur eingeschränkte Möglichkeiten zum gezielten Kompetenzerwerb: Fehlendes Textverständnis kann nur im Nachhinein durch Lehrende ergänzt werden und Hausarbeiten werden oft nur einmalig und flüchtig korrigiert.
Dringend benötigtes individuelles Feedback entfällt meist ganz, eine \emph{Deliberate Practice} \parencite{Ericsson2007} kann sich so kaum entfalten.
%VK: Kannst du bitte Deliberate Practice kurz definieren?

Ebenso leidet herkömmliche universitäre Lehre des akademischen Schreibens unter mangelnder Leserschaft, da die meisten Hausarbeiten im Orkus des universitären Vergessens verschwinden.
\footnote{
	\ldots auch wenn das vielleicht eine realistische Vorbereitung auf wissenschaftliches Arbeiten ist.
}
%VK: Sarkasmus, baby.
Studierende benötigen am Beginn des akademischen Schreibens aber gerade Anerkennungs- und Resonanzerfahrungen, auch um Entfremdung und Steigerungszwängen moderner Gesell\-schaften (etwa Burnout) präventiv entgegenzuwirken \parencite{Rosa-Paech-etal-2014}.

Schließlich gilt für Studierende der Sozial- und Geisteswissenschaften über ihr Fachstudium hinaus, generelle erwerbsrelevante und soziale Kompetenzen zu erwerben, wie etwa produktive Arbeiten in Gruppen oder der Umgang mit professioneller Software.
Nominelle Gruppenarbeiten werden oft von wenigen, oder einzelnen Gruppenmitgliedern erledigt (Trittbrettfahrereffekt), unter anderem wegen mangelnder Transparenz über geleistete Beiträge, Unklarheiten über den Modus der Zusammenarbeit oder aufgrund technischer Hindernisse im Datenaustausch.
Mit Ausnahme von Spezialsoftware der empirischen Sozialforschung und Statistik erfahren Studierende der Sozial- und Geisteswissenschaften selten umfangreiche Softwarekenntnisse.
In den Erziehungswissenschaften, besonders in der Lehrerbildung, ist die Skepsis bezüglich technischer Anwendungen erfahrungsgemäß recht hoch.
In beiden Fällen braucht es einen Anschub aus anderen Disziplinen.

Die Konventionen und Werkzeuge der \emph{Open Source} Softwareentwicklung bieten ein attraktives Modell, um intensive Textarbeit und resonanz-spendende Zusammenarbeit in der Fachdidaktik der Sozial- und Geisteswissenschaft neu zu beleben.
Wir konzentrieren uns hier auf die gegenwärtig dominierende (Open Source) Software \emph{git}, einem Werkzeug \emph{dezentraler} Versions- und Quellcodeverwaltung und auf die ebenfalls beliebte, darauf basierende (kommerziell-proprietäre) Plattform \emph{GitHub}, einem sozialen Netzwerk für Programmierer.

Wir erproben diese Technologie im Sommer 2014 in einem interdisziplinären Kurs über deliberative Demokratie und inklusive Pädagogik bei der Deutschen SchülerAkademie (Bildung \& Begabung gGmbH, Teil der Begabtenförderung des Bundesministeriums für Bildung und Forschung).

Die Zusammenarbeit mit \emph{git} und auf \emph{GitHub} zeichnet sich aus durch:
\begin{enumerate}
	\item Offenheit \& Erlaubnisfreiheit:
		%MH engl: permissiveness, laut dict erlaubtheit, sexuelle Freizügigkeit, toleranz
		\emph{Jeder} kann eigene Varianten (sogenannte \emph{Forks}, engl. Gabeln) von Projekten anlegen und nach eigenen Vorstellungen fortentwickeln.
		\emph{Jeder} kann den Fortgang von anderen Projekten und allen Forks nachvollziehen.
		\emph{Jeder} kann Probleme (\emph{Issues}) berichten, oder eigene Verbesserungsvorschläge für das gemeinsame Projekt unterbreiten (\emph{Pull Requests}).
	\item Effizienz \& Robustheit:
		Forks von verschiedenen Autoren, oder Versionen zu unterschiedlichen Zeitpunkte können Zeilen- und Buchstabengenau verglichen (\emph{Diff}) und gegebenenfalls teil-automatisiert vereint werden (\emph{Merge}).
		Bei korrektem Gebrauch können Daten kaum verloren gehen.
		Alle Versionen bleiben erhalten.
		Die zu verwendende Software stellt geringste Anforderungen an Hardware oder Netzanbindung.
	\item Wertschätzung \& Zurechenbarkeit:
		Rückmeldungen sind technisch elegant zu realisieren (auf Zeilenbasis), und werden durch Konventionen gestärkt (\emph{Code Review}).
		Verbesserungsvorschläge für andere Projekte sind einfach zu unterbreiten, und werden als bereichernd geschätzt (\emph{Pull Requests}, s.o.).
		Urheberschaft lässt sich Zeilengenau nachvollziehen (\emph{Blame}).
\end{enumerate}

Wenn möglich, sollte für das Erlernen der Software eine separate Einführung mit Übungsanteilen angeboten werden, sodass die Lernenden ausreichend Zeit haben, sich aktiv-entdeckend und an konkreten Problemen mit den technischen Abstraktionen auseinanderzusetzen und langsam die eigenen Schreibroutinen zu verändern.

Sofern basale Fähigkeiten im Umgang mit der Software erworben wurden, lässt sich beobachten, dass sowohl die individuelle Auseinandersetzung mit den Kursinhalten als auch kooperatives Lernen \textcite[vgl.][]{johnson-1999} gestärkt wurden.

\begin{enumerate}

	\item Die Teilnehmenden schreiben einen gemeinsamen Text.
	Sie sind insofern aufeinander angewiesen, alsdass das Lernziel nur in gemeinsamer Zusammenarbeit erreichten werden kann.
	Durch die Verteilung konkreter Teilaufgaben (Issues) verteilt sich die Verantwortung auf alle Schultern (\emph{positive Interdependenz}).
	\item Veränderungen am gemeinsamen Produkt sind für alle transparent.
	Jede Zeile, jeder Buchstabe und jedes Issue ist eindeutig einer Person zuzuordnen, sodass die \emph{Individuelle Verantwortlichkeit} hoch ist.
	\item \emph{Sich gegenseitig unterstützende Interaktionen} entstehen durch Feedback und Überarbeitungen von Textpassagen.
	\item Durch die Erlaubnisfreiheit, alles überall ändern zu können, sind die Teilnehmenden zu Selbstorganisation angehalten.
	Sie entwickeln \emph{soziale Fertigkeiten}.
	\cite[vgl.][]{johnson-1999}
	\item Zahlreiche Überarbeitungsdurchgänge des Textes und gegenseitiges Feedback erzeugen eine Lernumgebung im Sinne einer Deliberate Practice.
	%VK: Cite in barjoke_library.bib ergänzen (Ericsson 2007)

\end{enumerate}


%VK: DIE FOLGENDEN ABSCHNITTE WÜRDE ICH RAUSNEHMEN; FEHLT NOCH EIN FAZIT/TEXTENDE

% Positive interdependence. “Their work benefits us and our work benefits them” (Johnson & Johnson 1999: 71) :
% all students write one text
% students depend on another’s work on atomistic issues


% Individual accountability. “The performance of each individual student is assessed and the results are given back to the group” (ibid.) :
% work is transparent: everyone can see everything
% every line, word or issue posted is owned by individual students


% Promotive Interaction. “Individuals promote each other’s success by helping, assisting, supporting, encouraging and praising” (ibid.) :
% students give and receive mutual writing feedback
% students open and assign themselves or others to issues


% Social Skills & Group Processing. “Define and solve the problems they are having working together effectively” (ibid.):
% students learn conventions and technology to collaborate intensively
% students must organize themselves


% Deliberate Practice (Ericsson 2007):
% students write many revisions with intensive feedback

% %VK: Kritik?

% References:
% Ericsson, K. Anders (2007): “The Making of an Expert”, Harvard Business Review (July-August).

% Wir erhoffen uns von dieser Form der Zusammenarbeit einen zweifachen Lernerfolg:
% \begin{enumerate}
% 	\item \emph{Intensivierte Zusammenarbeit.}
% 	%VK Kooperatives Lernen
% 		Die offene Zusammenarbeit an \emph{einem} Text, mit vielen Revisionen und umfangreichen Rückmeldungen von Unterrichtenden und anderen Studierenden intensiviert die Einübung eines präzisen schriftlichen Ausdrucks im Sinne einer \emph{Deliberate Practice} \parencite{Ericsson2007}.
% 		Studierende können trotzdem ohne Ablenkung an ihren jeweils \emph{eigenen} Versionen arbeiten (\emph{dezentrale} Quellcodekontrolle), und diese später einpflegen.
% 		Des Weiteren hilft die minimalistische Benutzeroberfläche (reine Textdateien) den Studierenden sich auf das Schreiben zu konzentrieren.
% 	\item \emph{Reflexion \emph{über} die Zusammenarbeit.}
% 		Die selbstorganisierte (!) Zusammenarbeit mit oben erwähnten Konventionen zur Wertschätzung und Zurechenbarkeit bietet einen eindrucksvollen Betrachtungsgegenstand um über einige (wenn auch nicht alle) Konzepte der Sozial- und Geisteswissenschaft produktiv zu reflektieren.

% 		\emph{Open Source} ist gleichermaßen einer der kennzeichnenden (und vielleicht vielversprechensten) Modi der Zusammenarbeit und des Lernens unserer Zeit, also \emph{eine} Antwort auf Fragen der Soziologie und Pädagogik.
		%Wie plausibel \emph{kommunikative Rationalität} \cite{Habermas-1984} sein mag oder --- aus anderem Blickwinkel --- wie eine \emph{emergente Ordnung} \cite{hamowy_constitution_2011} wächst, können Studierende in der eigenen Auseinandersetzung mit verschiedenen \emph{Pull Requests}, \emph{Issues} über den gemeinsamen Text lebendig erfahren.
		%MH ähnlichen Beitrag über Pädagogik? Passt das?
	%\item Erwerbsrelevante Fertigkeiten.
	%	Schließlich lernen  Studierende mit \emph{git} ein gängiges Programm kennen, geraten in Kontakt mit Open Source Software und sind ermutigt intensiver über die eigene Nutzung von EDV und Organisation von digitaler Zusammenarbeit zu reflektieren.
		%MH hier später mal noch ein beef über "software is eating the world" einfügen
%\end{enumerate}

%\item[Inhaltlich-fachbezogene Ziele.]
%	Selbstverständlich geht es im Studium auch um das Studien\emph{fach}, hier etwa um Abstraktionen und Konzepte der Geistes- und Sozialwissenschaften,

		%VK: Würde ich weglassen, zu detailliert
		%wie zum Beispiel die \emph{Reformpädagogik} \cite{Freinet1979}
					%MH korrektes Zitat und Beschreibung?
				%oder \emph{kommunikative Rationalität} \cite{Habermas1988a}.
					%MH die citebefehle in klammern haben sich geändert in biblatex, das muss ich noch ändern.]
		%Auch hier steht die Fachdidaktik vor Problemen.
%		So fällt es oft schwer, bei Studierenden Interesse für abstrakte Literaturen zu wecken --- jedenfalls ohne die betreffenden Inhalte zu trivialisieren.

		%VK: Annahme?
		%Konzepte wie etwa \emph{Reformpädagogik} oder \emph{kommunikative Rationalität} bieten keinen unmittelbaren Sinneseindruck oder präzisen Formalismus (wie etwa viele naturwissenschaftliche Gegenstände), und entbehren so für Studierende leicht eines Zusammenhanges zu ihrer Lebenswelt.

%		Dieses didaktische Scheitern produziert nicht nur motivationale Probleme, sondern verletzt auch den emanzipatorischen Ethos einer humanistischen Bildung im allgemeinen %MH Zitat irgendwas?
%		oder öffentlichen Sozialwissenschaft im besonderen \cite{gans1989public-sociology}.
%		Schließlich schlägt sich mangelndes Verständnis auch in fehlschlagenden Transferleistungen und schnellem Vergessen der Lerninhalte wieder, wie etwa \cite{Frank2007a} anektodisch über Einführungen in die Volkswirtschaftslehre berichtet.
%		Damit Studierende nicht nur definitorisches Wissen reproduzieren (sogenanntes ``bulimisches Lernen''), braucht es gute Anwendungsfälle um die bearbeiteten Konzepte erfahrbar zu machen.

%Max notes from making keynote
% deliberate practice
% learning by researching
% resonanzerfarung
% lernumgebumg
% cooperative learning
% transparenz, robustheit, einfachheit



\printbibliography

\end{document}

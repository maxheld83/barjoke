\documentclass[11pt,a4paper,oneside]{article}
\usepackage{./latexstyles/print}
\selectlanguage{ngerman}
\addbibresource{../held-library/held_library.bib}
\addbibresource{../vk_library/verenakasztantowicz.bib}

\title{
	Kooperative Schreiben mit GitHub ---
	Ein wissenschaftsdidaktisches Experiment mit TeilnehmerInnen des Kurses ``Schule und Demokratie'' der Deutschen SchülerAkademie
}
%MH: vielleicht doch Variante des albernen/sexieren Titels "treffen sich ..... auf github nehmen"? Oder bewahren wir uns das für später auf?

\date{May 31, 2014}

\author{
	Maximilian Held\\
	Bremen International Graduate School of Social Sciences (BIGSSS)
	\and
	Verena Kasztantowicz\\
	Grundschulpädagogik, Humboldt-Universität zu Berlin (HU)
}

\begin{document}

\maketitle

%\abstract{

%}

%zu berücksichtende stichwörter aus der Ausschreibung (buzzwords)
	%forschendes lernen,
	%peer-learning,
	%interdisziplinäre lehre

Akademische Lehre der Geistes- und Sozialwissenschaften verfolgt
\begin{inparaenum}
	\item methodisch-formale,
	\item inhaltlich-fachbezogene sowie
	\item praktisch-erwerbsrelevante
\end{inparaenum}
Lernziele.
	%MH: gibt's dafür vielleicht ne Quelle? Macht das Sinn?
In allen drei Bereichen sieht sich die Didaktik der Geistes- und Sozialwissenschaften alten Problemen und neuen Herausforderungen gegenüber.

\begin{description}
	\item[Methodisch-formale Ziele.]
		Studierende der Geistes- und Sozialwissenschaften benötigen umfangreiche Medien- und Methodenkompetenzen: Um selbstständig wissenschaftlich zu arbeiten, müssen sie etwa Literaturrecherchen unternehmen, Fachtexte kritisch lesen und eigene Arbeiten zu Papier bringen.
		Besonders wissenschaftliches Lesen und Schreiben stellt Studierende (vielleicht zunehmend) vor Probleme.
		%MH: finden wir hierfür noch eine Literatur?
		Studierende sind oft von wissenschaftlicher Literatur überfordert, und abgegebene Texte genügen oft nicht den Erwartungen.
		Herkömmliche universitäre Lehre, ob als Seminar oder Vorlesung, bietet hier traditionell nur eingeschränkte Möglichkeiten zum intensiven Lernen: fehlendes Textverständnis kann nur im Nachhinein durch Lehrende ergänzt werden, und Hausarbeiten werden oft nur einmalig und flüchtig korrigiert.
		Sowohl kritische Lesegewohnheiten als auch präziser Schriftausdruck benötigen aber \emph{deliberate practice} \cite{Ericsson2007}, also vielfache Wiederholung begleitet von sorgfältigen Rückmeldungen.
		Um gut Lesen und Schreiben zu lernen, brauchen Studierende zahlreiche, kleinteilige Aufgaben, die auf Resonanz stoßen.
		%MH: hier tatsächlich irgendwo Hartmut Rosa oder dingens der Vörgänger, wie heisst er ... re: resonanz einbinden? Können wir über Resonanz noch etwas sagen?
	\item[Inhaltlich-fachbezogene Ziele.]
		Selbstverständlich geht es im Studium auch um das Studien\emph{fach}, hier etwa um Abstraktionen und Konzepte der Geistes- und Sozialwissenschaften, wie zum Beispiel die \emph{Reformpädagogik} \cite{Freinet1979}
			%MH korrektes Zitat und Beschreibung?
		oder \emph{kommunikative Rationalität} \cite{Habermas1988a}.
			%MH die citebefehle in klammern haben sich geändert in biblatex, das muss ich noch ändern.
		Auch hier steht die Fachdidaktik vor Problemen.
		So fällt es oft schwer, bei Studierenden Interesse für abstrakte Literaturen zu wecken --- jedenfalls ohne die betreffenden Inhalte zu trivialisieren.
		Konzepte wie etwa \emph{Reformpädagogik} oder \emph{kommunikative Rationalität} bieten keinen unmittelbaren Sinneseindruck oder präzisen Formalismus (wie etwa viele naturwissenschaftliche Gegenstände), und entbehren so für Studierende leicht eines Zusammenhanges zu ihrer Lebenswelt.
		Dieses didaktische Scheitern produziert nicht nur motivationale Probleme, sondern verletzt auch den emanzipatorischen Ethos einer humanistischen Bildung im allgemeinen %MH Zitat irgendwas?
		oder öffentlichen Sozialwissenschaft im besonderen \cite{gans1989public-sociology}.
		Schließlich schlägt sich mangelndes Verständnis auch in fehlschlagenden Transferleistungen und schnellem Vergessen der Lerninhalte wieder, wie etwa \cite{Frank2007a} anektodisch über Einführungen in die Volkswirtschaftslehre berichtet.
		Damit Studierende nicht nur definitorisches Wissen reproduzieren (sogenanntes ``bulimisches Lernen''), braucht es gute Anwendungsfälle um die bearbeiteten Konzepte erfahrbar zu machen.
	\item[Erwerbsrelevante Fertigkeiten.]
		Schließlich gilt es für Studierende der Sozial- und Geisteswissenschaften über ihr Fachstudium hinaus generelle erwerbsrelevante Fertigkeiten zu erwerben, wie etwa gutes Arbeiten in Gruppen oder Umgang mit professioneller Software.
		Auch in diesem Bereich gibt es aus eigener Erfahrung in unseren Fachgebieten einiges nachzuholen.
		Nominelle Gruppenarbeiten werden oft von wenigen, oder einzelnen Gruppenmitgliedern erledigt, unter anderem wegen mangelnder Transparenz über geleistete Beiträge, Unklarheiten über den Modus der Zusammenarbeit oder technischen Hindernissen im Datenaustausch.
		Mit Ausnahme von Spezialsoftware der empirischen Sozialforschung und Statistik glänzen Studierende der Sozial- und Geisteswissenschaften oft auch nicht mit umfangreichen Softwarekenntnissen.
		Hier braucht es einen Anschub aus anderen Disziplinen.
\end{description}

Die Konventionen und Werkzeuge der \emph{Open Source} Softwareentwicklung bieten ein attraktives Modell der Zusammenarbeit, um diesen alten Probleme und neuen Herausforderungen der sozial- und geisteswissenschaftlichen Hochschullehre neu zu begegnen.
%MH Open source und git eventuell anders setzen, monospace oder so.
Wir konzentrieren uns hier auf die gegenwärtig dominierende  (Open Source) Software \emph{git}, einem Werkzeug \emph{dezentralen} Versions- und Quellcodeverwaltung und auf die ebenfalls beliebte, darauf basierende (kommerziell-proprieträre) Plattform \emph{GitHub}, einem sozialen Netzwerk für Programmierer.

Die Zusammenarbeit mit \emph{git} und auf \emph{GitHub} zeichnet sich aus durch:
\begin{enumerate}
	\item Offenheit & Erlaubnisfreiheit:
		%MH engl: permissiveness, laut dict erlaubtheit, sexuelle Freizügigkeit, toleranz
		\emph{Jeder} kann eigene Varianten (sogenannte \emph{Forks}, engl. Gabeln) von Projekten anlegen und nach eigenen Vorstellungen fortentwickeln.
		\emph{Jeder} kann den Fortgang von anderen Projekten und allen Forks nachvollziehen.
		\emph{Jeder} kann Probleme (\emph{Issues}) berichten, oder eigene Verbesserungsvorschläge unterbreiten (\emph{Pull Requests}).
	\item Effizienz & Robustheit:
		Forks von verschiedenen Autoren, oder Versionen zu unterschiedlichen Zeitpunkte können elegant Zeilen- und Buchstabengenau verglichen (\emph{Diff}) und gegebenfalls teil-automatisiert vereint werden (\emph{Merge}).
		Bei korrektem Gebrauch können Daten kaum verloren gehen; alle Versionen bleiben erhalten.
		Die zu verwendende Software stellt geringste Anforderungen an Hardware oder Netzanbindung.
	\item Wertschätzung & Zurechenbarkeit:
		Rückmeldungen sind technisch elegant zu realisieren (auf Zeilenbasis), und werden durch Konventionen gestärkt (\emph{Code Review}).
		Verbesserungsvorschläge für andere Projekte sind einfach zu unterbreiten, und werden als bereichernd geschätzt (\emph{Pull Requests}, s.o.).
		Urheberschaft lässt sich Zeilengenau nachvollziehen (\emph{Blame}).
\end{enumerate}

Diese Konventionen und Werkzeuge ermöglichen auch für Sozial- und Geisteswissenschaftliches Lernen eine intensive Zusammenarbeit, wie sie herkömmlich (etwa mit binären Dateiformaten wie MS Office oder zentralen Speichern wie Dropbox) nicht möglich ist.

Wir erproben diese Technologie im Sommer 2014 in einem Kurs über deliberative Demokratie und inklusive Pädagogik bei der Deutschen SchülerAkademie (Bildung & Begabung gGmbH).
Teilnehmende (15-18 Jahre alt) werden auf GitHub gemeinsam einen etwa 30-seitigen Text (sogenannte ``Dokumentation'') über den Kursinhalt verfassen.

Wir erhoffen uns von dieser (anders nicht möglichen) intensivierten Zusammenarbeit Lernerträge im Sinne aller drei oben genannten Ziele:
\begin{description}
	\item[Methodisch-formale Ziele.]
		Die offene Zusammenarbeit an \emph{einem} Text, mit vielen Revisionen und umfangreichen Rückmeldungen von Unterrichtenden und anderen Studierenden intensiviert die Einübung eines präzisen schriftlichen Ausdrucks im Sinne einer \emph{Deliberate Practice} \cite{Ericsson2007}.
		Studierende können trotzdem ohne Ablenkung an ihren jeweils \emph{eigenen} Versionen arbeiten (\emph{dezentrale} Quellcodekontrolle), und diese später einpflegen.
		Des weiteren hilft die minimalistische Benutzeroberfläche (reine Textdateien) den Studierenden sich auf das Schreiben zu konzentrieren.
	\item[Inhaltlich-fachbezogene Ziele.]
		Die selbstorganisierte (!) Zusammenarbeit mit oben erwähnten Konventionen zur Wertschätzung und Zurechenbarkeit bietet einen eindrucksvollen Betrachtungsgegenstand um über einige (wenn auch nicht alle) Konzepte der Sozial- und Geisteswissenschaft produktiv zu reflektieren.
		\emph{Open Source} ist gleichermaßen einer der kennzeichnenden (und vielleicht vielversprechensten) Modi der Zusammenarbeit und des Lernens unserer Zeit, also \emph{eine} Antwort auf Fragen der Soziologie und Pädagogik.
		Wie plausibel \emph{kommunikative Rationalität} \cite{Habermas-1984} sein mag oder --- aus anderem Blickwinkel --- wie eine \emph{emergente Ordnung} \cite{hamowy_constitution_2011} wächst, können Studierende in der eigenen Auseinandersetzung mit verschiedenen \emph{Pull Requests}, \emph{Issues} über den gemeinsamen Text lebendig erfahren.
		%MH ähnlichen Beitrag über Pädagogik? Passt das?
	\item[Erwerbsrelevante Fertigkeiten.]
		Schließlich lernen  Studierende mit \emph{git} ein gängiges Programm kennen, geraten in Kontakt mit Open Source Software und sind ermutigt intensiver über die eigene Nutzung von EDV und Organisation von digitaler Zusammenarbeit zu reflektieren.
\end{description}

%MH hier Diagramm einfügen über die Übereinstimmung der drei Ziele? Sozusagen Venn-Diagramm mit GitHub im Schnittpunkt?

%MH hier später mal in einer großen Tabelle zusammentragen wie die *ziele* (spalten) und die *besonderheiten* (zeilen) von GitHub zusammenspielen.

Im Rahmen der Deutschen SchülerAkademie wird im Sommer 2014 ein interdisziplinärer Kurs zum Thema ``Schule und Demokratie'' unter unserer Leitung stattfinden.
Zielgruppe sind SchülerInnen im Altern von 16 bis 18 Jahren, die den 3-wöchigen Kurs im Rahmen des Begabtenförderungsprogramms gewählt haben.

\section{Inhaltliche Konzeption des Kurses ``Schule und Demokratie''}

Inhaltlich steht im Zentrum des Kurses ``Schule und Demokratie'' der Widerspruch und die wechselseitige Bedingtheit von inhärenter Gleichheit und Autonomie im menschlichen Zusammenleben \cite{Habermas1999a}.

Ein konstitutiver Konflikt, der sowohl in der Pädagogik als auch in der politischen Theorie anhaltend diskutiert wird.
Es geht dabei einerseits um die Frage, wie Schule, wie Lernen, und letztlich auch menschliche Beziehungen unter der Prämisse von unterschiedlichen Voraussetzungen demokratisch gestaltet werden kann.
Andererseits wird nach der Legitimität kollektiv verbindlicher Entscheidungen in Politik und Gesellschaft gefragt.
Unter dieser Perspektive rücken auch aktuelle Ansätze wie Formen deliberativer Demokratie (Cohen 1989) und inklusiver Pädagogik (Eberwein/Knauer 2009) ins Blickfeld und sollen innerhalb des Kurses schwerpunktmäßig behandelt werden.
Abschließend versucht der Kurs eine radikale Synthese der beiden Bezugsdisziplinen Pädagogik und politischer Theorie, nämlich, dass eine demokratische Schule eine Schule für alle Kinder und Jugendlichen sein müsse im Sinne inklusiver Pädagogik (Feuser 2001) und deliberative Demokratie gleichermaßen der operativen Metapher der Schule folgen solle (Rosenberg 2002).

\section{Wissenschaftsdidaktische Konzeption des Kurses ``Schule und Demokratie''}

Die beiden vertretenen Disziplinen Pädagogik und politische Theorie erfordern nicht nur weitgreifende inhaltliche sondern auch didaktische Brückenschläge, die mithilfe von innovativen Technologien gestalten werden sollen.

Die Basis unserer Kursarbeit --- und man könnte auch behaupten unserer Disziplinen --- ist der Text, das zusammenhängende geschriebene Wort.
Dabei wird die These vertreten, dass nicht allein durch die Textrezeption, sondern erst durch produktiven und kommunikativen Umgang mit der Literatur ein tiefgreifendes Verständnis bei den TeilnehmerInnen angebahnt wird.
So ist auch das Ziel des Kurses *ein* Text, in Form einer gemeinsam geschriebenen Dokumentation, welche abschließend mit den anderen Beiträgen der Akademie veröffentlicht wird.
Ein Text schreibt sich bekanntlich nicht *von allein*, aber am Besten doch allein, so die landläufige Meinung.
Wie kann trotzdem partizipativ an einem Text geschrieben werden?
Wie bekommen trotzdem alle 15 TeilnehmerInnen die Möglichkeit, sich mit Ideen zu inhaltlichen Schwerpunkten, zum Aufbau und der Form der Dokumentation einzubringen?
Diese Fragen, die normalerweise zu verheerenden E-Mailbombardements führen oder unüberschaubare .doc-Sammlungen in den Dropboxen dieser Welt produzieren, haben gerade für die textroutiniertesten Disziplinen, der Geistes- und Sozialwissenschaften, eine besondere Dringlichkeit.
Damit das Chaos nicht vorprogrammiert ist, haben wir uns an einer weiteren Disziplin orientiert, die tagtäglich vor den Problemen von Textproduktion und Zusammenarbeit steht: Programmierer.
In der Informatik werden seit Langem Plattformen zur Herstellung von Software genutzt, Beispiel hierfür sind ...
    % weitere Anbieter?
Wir haben uns für GitHub entschieden, ein kommerzieller Anbieter, der allerdings für den Bildungskontext freie Kapazitäten zur Verfügung stellt.
    % technische Details und Ablauf auf GitHub

Vorteile der Benutzung dieser Software ist die lückenlose digitale Dokumentation und damit auch die Sicherung aller Erarbeitungsschritte und Ergebnisse; die Möglichkeit aller TeilnehmerInnen direkt auf Textstellen (Absätze, Sätze, einzelne Wörter) Bezug zu nehmen, zu kommentieren und Veränderungen direkt im Text vorzunehmen, die von den anderen Autoren wiederum akzeptiert oder abgelehnt werden können.
    % weitere Vorteile

Nur so kann der Kurs selbst seinem inhaltlichen Postulat nach demokratisch legitimen Zusammenwirken innerhalb der Gemeinschaft des Kurses gerecht werden.


\printbibliography

\end{document}